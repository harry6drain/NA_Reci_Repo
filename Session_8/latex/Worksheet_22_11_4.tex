\documentclass[11pt,letterpaper]{article}
\usepackage[utf8]{inputenc}
\usepackage[left=1in,right=1in,top=1in,bottom=1in]{geometry}
\usepackage{amsfonts,amsmath}
\usepackage{graphicx,float}
% -----------------------------------
\usepackage{hyperref}
\hypersetup{%
  colorlinks=true,
  linkcolor=blue,
  citecolor=blue,
  urlcolor=blue,
  linkbordercolor={0 0 1}
}
% -----------------------------------
\usepackage{fancyhdr}
\newcommand\course{MATH-UA.0252\\Numerical Analysis}
\newcommand\hwnumber{8}                  % <-- homework number
\newcommand\NetIDa{Ryan Sh\`iji\'e D\`u} 
\newcommand\NetIDb{November 4th, 2022}
\pagestyle{fancyplain}
\headheight 35pt
\lhead{\NetIDa\\\NetIDb}
\chead{\textbf{\Large Worksheet \hwnumber}}
\rhead{\course}
\lfoot{}
\cfoot{}
\rfoot{\small\thepage}
\headsep 1.5em
% -----------------------------------
\usepackage{titlesec}
\renewcommand\thesubsection{(\arabic{section}.\alph{subsection})}
\titleformat{\subsection}[runin]
        {\normalfont\bfseries}
        {\thesubsection}% the label and number
        {0.5em}% space between label/number and subsection title
        {}% formatting commands applied just to subsection title
        []% punctuation or other commands following subsection title
% -----------------------------------
\setlength{\parindent}{0.0in}
\setlength{\parskip}{0.1in}
% -----------------------------------
\input{../../command.tex}
\begin{document}

\section{QR decomposition via Householder}
\subsection{}
Construct the QR factorization of the following matrix
  using Householder reflectors:
\begin{align*}
{A} = 
\begin{bmatrix}
1 & 0 & 1 \\
2 & 1 & 3 \\
0 & 2 & 4
\end{bmatrix}.
\end{align*}
\subsection{}
Use the factorization to determine $|\det(A)|$

\section{Gershgorin disks and the power method}
Consider the matrix 
$$
A = \begin{bmatrix}
- 6 & 2 & 0.3 & 0 & -0.7 \\
2 & - 4 & 0.1 & 0.05 & 0 \\
0.3 & 0.1 & 2 & 0.1 & 0.1 \\
0 & 0.05 & 0.1 &  4 & 0 \\
-0.7 & 0 & 0.1  & 0  & 6
\end{bmatrix}
$$
and recall the definition of the Gershgorin disks:
\[
D_i = \{ z \in \mathbb C ~|~ |z - a_{ii}| \le \sum_{j \ne i} |a_{ij}| \}.
\]

\subsection{}
Argue that all eigenvalues of $A$ are real.

\subsection{}
What are the Gershgorin disks for $A$?  Use them to give a set, $D \subset \mathbb{R}$, that contains all eigenvalues of $A$.

\subsection{}
Can you conclude that the eigenvalue with the largest absolute value is simple?

\subsection{}
Argue that $A$ is invertible. Conclude that all diagonally dominant matrix is invertible.

\subsection{}
True or False? Let $A \in \mathbb{R}^{n \times n}$ and $D_i$, $i
  = 1,2,\dots,n$, be the Gerschgorin disks of $A$. If $0 \in
  \bigcup_{i=1}^n D_i$ then $A$ is singular.
  
\subsection{}
Write down the first iteration of the power method starting from $\ve x_0 = (0,0,0,0,1)^T$.
You don't need to normalize.
Explain why $\ve x_0 = \ve 0$ is not a suitable starting point.

\subsection{}
The eigenvalues of $A$, after rounding, are $\{-7, -3, 2, 4, 6\}$.
Which eigenvalue direction will the sequence of the previous question converge to?

\section{Eigenvectors as stationary points of Rayleigh quotient}
% Let $H$ be a real symmetric matrix. Denote the eigenvalues of $H$, arranged in increasing order, by $a_1,\dots,a_n$. Then we have that
% \begin{align*}
%     a_j = \min_{\dim S = j}\max_{x\in S,x\neq 0} \frac{x^\top Hx}{x^\top x}
% \end{align*}
% where $S$ is a linear subspace of $X$ \cite[p.116]{Lax_07}.

For $H$ a real symmetric matrix, we define Rayleigh quotient as a function $\mathbb{R}^n\to\mathbb{R}$:
\begin{align*}
    R(\ve x) = \frac{\ve x^\top H\ve x}{\ve x^\top \ve x} = \frac{q(\ve x)}{p(\ve x)}.
\end{align*}
We will show that $\ve v$ is a stationary point (i.e.: $\nabla R(\ve v) = 0$) of the Rayleigh quotient if and only if it is an eigenvector of $H$ (cf. \cite[p.114-116]{Lax_07} and \cite[p.203-204]{TrefethenBau_97}).

\subsection{}
To characterize a point such that $\nabla R(\ve v) = 0$, we need to know the gradient of $R(\ve x)$ at $\ve v$. We could do this, but an alternative approach is to take $t\in \mathbb{R}$ and calculate
\begin{align*}
    \left.\frac{\de}{\de t} R(\ve v+t\ve y)\right|_{t=0}
\end{align*}
for all $\ve y\in\mathbb{R}^n$. In particular, we can get the gradient by picking $\ve y = \ve e_i$.

\subsection{}
Using the above calculation, shows the iff claim in the main text of the problem.


\vfill
\bibliographystyle{alpha}
\bibliography{citation}

\end{document}